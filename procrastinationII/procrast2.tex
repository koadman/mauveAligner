%
% string matching writeup...
%

\documentclass{llncs}
\pagestyle{empty}
\usepackage{makeidx}  % allows for indexgeneration
%
\usepackage[dvips]{graphicx}    % needed for including graphics e.g. EPS, PS
\usepackage{epsfig}
\usepackage{url}
\usepackage{pseudocode}
\usepackage{comment}
\usepackage{authblk}
\usepackage{color}


\begin{document}
\renewcommand{\labelenumi}{(\Alph{enumi})}
\renewcommand{\labelenumii}{(\alph{enumii})}

\frontmatter          % for the preliminaries
%
\pagestyle{headings}  % switches on printing of running heads

\mainmatter              % start of the contributions
%
\title{procrastination II: beyond filtration}
%
\titlerunning{procrastination for local multiple alignment}  % abbreviated title (for running head)
%                                     also used for the TOC unless
%                                     \toctitle is used
\author{Don Quijote}
%
\authorrunning{<Quijote> et al.}   % abbreviated author list (for running head)
\institute{ Castilla La Mancha, Espa\~na }
%
%\institute{ Dept. of Computer Science,
%Univ. of Wisconsin-Madison, USA\\
%\email{darling@cs.wisc.edu},\\
%\and Dept. of Computer Science, Technical Univ. of Catalonia, Barcelona, Spain\\
%\email{treangen@lsi.upc.edu},\\


\maketitle


\begin{abstract}
We describe an efficient local multiple alignment
heuristic for the sensitive identification of
conserved and repetitive regions in one or more DNA
sequences.  The GPL implementation of our algorithm in C++ is
called \texttt{procrastAligner} and is freely available from
\url{http://gel.ahabs.wisc.edu/procrastination}
\end{abstract}




\section{ Introduction }

\subsection{Brief history \& overview of sequence alignment}
\begin{itemize}

\item Pairwise local sequence alignment has a played a central role 
in computational biology and new approaches continue to be
proposed~\cite{ref-pattern,ref-chaos,ref-yass,ref-kahveciMAP}.

\item Local alignment can be used for orthology
mapping~\cite{ref-orthomcl}, genome assembly~\cite{ref-arachne2}, repeat identification~\cite{...},
and information engineering tasks such as data
compression~\cite{ref-ane}.

\item Recent advances in sequence data
acquisition technology~\cite{ref-454} provide low-cost sequencing
will continue to grow molecular sequence
databases.

%motivation
\item To cope with such increases in data volume, corresponding
advances in computational methods are necessary; thus we present
an efficient method for de novo detection of homologous regions in multiple DNA sequences.

\end{itemize}

\subsection{why multiple alignment?}
\begin{itemize}

\item Unlike pairwise alignment, local multiple alignment constructs a
single multiple alignment for all occurrences of a motif in one or
more sequences.  The motif occurrences may be identical or have
degeneracy in the form of mismatches and indels.

\item local multiple alignments identify the basic repeating units in one or
more sequences and can serve as a basis for downstream analysis
tasks such as multiple genome
alignment~\cite{ref-mauve,ref-mga,ref-mgcat,ref-deweyReview}, global
alignment with repeats~\cite{ref-otherSammethPaper,ref-aba}, or
repeat classification and analysis~\cite{ref-piler}.

\item Local multiple
alignment differs from traditional pairwise methods for repeat
analysis which either identify repeat families \textit{de
novo}~\cite{ref-reputer} or using a database of known repeat
motifs~\cite{ref-repbase}.

\end{itemize}

\subsection{related work on local multiple alignment}
\begin{itemize}

\item Eulerian path approach proposed by Zhang and Waterman~\cite{ref-related1}.

\item ABA, A-bruijn multiple alignment proposed by Raphael et al~\cite{ref-aba}.

%\item RepeatGluer, Pevzner et al~\cite{ref-repeatgluer}.


\end{itemize}
\subsection{what we offer/why we are different}
\begin{itemize}

%the main idea
\item Spaced Seeds$+$Procrastination$+$Extension$+$Simulations

\end{itemize}
\section{Overview of the Method}\label{sec:overview}

\begin{enumerate}
\item[I.] Find seeds: Find all (families of?) spaced k-mer seeds M in sequence S
%we now have matches
\item[II.] Run Evil megaloop: For all seeds $m_{i}$ in M
\begin{enumerate}
\item process highest multiplicity seeds/matches first
\item procrastinate lower multiplicity seeds/matches
\item chain seeds within w nucleotides
%we now have chained matches
\item performed gapped alignment on regions between chained seeds
\item If $c_{i}$ score is greater than extension threshold, trigger gapped extensions
\begin{enumerate}
\item perform gapped alignment on region spanning 3*w nucleotides to the left/right
\item call detectAndApplyBackbone on the left/right region to find the high scoring segments
\item perform pairwise homology prediction to unalign any non-homologous sequence
\item if boundaries are improved, extend to the left/right, try extending to the left/right again
\item else stop extending to the left/right
\end{enumerate}
\end{enumerate}
%we now have local multiple alignments
\item[IV.] Score local multiple alignments: for $l_{i}$ in l, score alignment using Sum of Pairs
\item[V.] Generate pvalue from Sum of Pairs score, taking multiplicity into account
\end{enumerate}


\subsection{Seed detection}
\begin{itemize}

\item palindromic spaced seeds, use seed families instead?

\item Begins by generating a set of candidate multi-matches using \textit{palindromic} spaced
seed patterns (~\cite{ref-procrast}).

\item The use of
\textit{palindromic} seed patterns offers computational savings by
allowing both strands of DNA to be processed simultaneously.

\item Could YASS be of any use?

\end{itemize}

\subsection{Seed chaining and procrastination}
\begin{itemize}
\item briefly describe and review original procrastAlign algorithm
\item point out that this current work is natural "extension" of previous work.
\end{itemize}
\subsection{Gapped Extension}
\begin{itemize}
\item After finding \& chaining seeds in input sequence, we would like to perform gapped
extension to detect all surrounding homologous sequence.
\end{itemize}
\subsubsection{Gapped alignment via MUSCLE}
\begin{itemize}
\item we employ MUSCLE to perform gapped alignment surrounding the chains

\item we trigger gapped alignment if chain is above some minimum score threshold.
idea is that extension is expensive, and so we need to minimize the number
of gapped extensions that do not improve the boundaries of the chain.

\item we grab 3*w nucleotides to the left/right of each chain. don't go past
chain component boundaries, though.

\item we score the resulting gapped alignment via sum of pairs(computeSPScore)

\item if we had a way to account for gaps in consensus, we could also
score this with the consensus sequence(see RepeatScout~\cite{ref-repeatscout})
\item W we extend to
the left(or right) until we reach a score(calculated by sum of pairs)
that has been previously determined via simulations to be
non-homologous. i.e. a peak in the score signals the start of a stretch of non-homologous sequence.  
\item If we improve our original seed boundaries we then trigger
another round of extension, else we stop. 
\item We also perform pairwise
homology predictions to make sure that our final local multiple
alignments do not have any non-homologous sequence inside of the
alignment, for all rows in each column. 
\item We
unalign and non-homologous sequence inside of our multiple alignments.



\end{itemize}

\subsubsection{Random walk to extend homology border}

\begin{itemize}
\item after scoring each column of the gapped extension, we would like
to determine where the homology ends/begins in this region
\item we do this by performing a random walk of the cumulative column score
\item we stop when we reach the record height
\item we determine the record height by simulations
\item all high scoring segments in the gapped extension are reported
\item if boundaries of the original chain are improved, trigger another round of extension.
\end{itemize}

\subsubsection{Processing Novel homologous sequence}
\begin{itemize}
\item novel homologous sequence can be found during gapped extension
\item should briefly describe what we do with this
\end{itemize}
\subsubsection{Define boundaries with pairwise homology predictions}
\begin{itemize}
\item our extension is not quite finished yet. Everything ok so far except that we could
still be including non-homologous sequence in our extended chain(local multiple alignment).
\item Define sets/subsets/boundaries with a pairwise homology test
\item first, we unalign any non-homolgous sequence introduced during the extension process
\item then, we define common boundaries among all sequences in the alignment
\end{itemize}
\subsection{local multiple alignment scoring}
\begin{itemize}
\item finally, we get a final score of our left/right extended chains using sum of pairs
\item keep track of multiplicity, length(?)
\end{itemize}
\subsection{local multiple alignment significance via simulations}
basically want to extend blast statistics to multiple local alignment. How?
\begin{itemize}
\item Extending Repseek simulations to multiple
\item Tompa et al, estimating the parameters for multiple alignments
\item will involve running procrastAlign with fixed parameters on sequences varying in composition and length
\item also will have to take into account multiplicity
\item FIXME: fill in more details on Monday
\end{itemize}

\subsection{Output}
\begin{itemize}
\item Ideally by now we will have a list of multiple local alignments, ordered by pvalue
\item store local multiple alignments in space efficient manner. e.g. A-bruijn graph, output a given column in the alignments once and only once.
\end{itemize}
\subsection{Limitations}
%limitations
\begin{itemize}
\item Not really useful for detecting subtle motifs such as transcription factor
binding sites in small, targeted sequence regions
\item Memory bottleneck for large sequences
\item palindromic spaced seeds not ideal for more diverged sequence. 
\end{itemize}
\subsection{Anything else?}
\begin{itemize}
\item we can use downstream methods to detect mosaic structure of consensus repeat families, breaks
    families into subset domains
    
\item visualization and/or incorporation into global/glocal multiple alignment tools 
\end{itemize}

\section{Results}
We have created a program called \texttt{procrastAligner} for Linux,
Windows, and Mac OS X that implements the described algorithm. Our
open-source implementation is available as C++ source code licensed
under the GPL.

\subsection{Comparison with previous version of procrastAligner I}
\begin{itemize}

\item Hepatitis C case, show improved sensitivity.
\item Alu repeats(show as good as previous version)
 
\end{itemize}

 \subsection{Comparison with RepeatScount}
\begin{itemize}
\item Number and total length of repeat families in our RepeatScout
library versus Repbase Update (Jurka, 1998, 2000) for human, mouse
and rat. Taken from Table 2 from paper.
\end{itemize}

\subsection{Run on new data}
 \begin{itemize}
 \item Ocean metagenomic data(6.4 GB)
 \item Detecting large segmental duplications(dispersed \& tandem)
\end{itemize}

\section{Discussion}

\section{ Acknowledgments }
AED was supported by NLM Training Grant 5T15LM007359-05. TJT was
supported by Spanish Ministry MECD Grant TIN2004-03382 and AGAUR
Training Grant FI-IQUC-2005.

\section{Figure legends}
\subsection{Figure 1: Visualization of Extension process}
\subsection{Figure 2: Example alignment}
\subsection{Figure 3: Main algorithm, pseudocode}

\section{Table legends}
\subsection{Table 1: Comparison with RepeatScout}
\subsection{Table 2: Comparison with procrastAlign I on Alu data, Hepatatitis C data}
\subsection{Table 3: ...}

\bibliographystyle{splncs}
\small
\bibliography{procrastination}


\end{document}
