%\documentclass[draft]{ws-procs9x6}
\documentclass{llncs}
\usepackage{color}
\usepackage{epsfig}
\usepackage{times}
\usepackage{url}
\begin{document}
\chapter*{Supplementary Material}
\section*{Calculating the Chiaromonte \textit{et al.} scaling factor}
The emission probabilities for
each possible pair of aligned nucleotides were extracted from the HOXD
substitution matrix presented by Chiaromonte \textit{et al}\cite{hoxd}.
We solved for the emission frequencies in the
Homologous and Unrelated state using the same equation used to
calculate the values of the HOXD substitution matrix on 47.5\%G+C
content sequence\cite{hoxd}:
\begin{equation}
s(x,y)= \log_{2}{\Bigg(\frac{p(x,y)}{q_{1}(x)q_{2}(y)}\Bigg)}
\end{equation}
{w}here $p(x,y)$ is the fraction of the observed aligned pairs of
nucleotides $x$ and $y$ in the training set used and $q_{1}(x)$ and
$q_{2}(x)$ denote the background frequencies of $x$ and $y$,
respectively. Chiaramonte \textit{et al.} scaled the resulting
$s(x,y)$ values by an unreported value $\psi$ so the largest was 100,
with the rest rounded to the nearest integer.  Given that the training
data has $47.5\%$~$G+C$ content and considering strand and species
symmetry, we can compute emission frequencies for the Unrelated state
of our HMM:
\begin{center}$U_{AA}=U_{AT}=U_{TA}=U_{TT}=(\frac{f_{AT}}{2})(\frac{f_{AT}}{2})
= 0.06890625$ \\
$U_{CC}=U_{CG}=U_{GC}=U_{GG}=(\frac{f_{GC}}{2})(\frac{f_{GC}}{2}) =
0.05640625$ \\
$U_{AC}=U_{AG}=U_{TC}=U_{AG}=(\frac{f_{AT}}{2})(\frac{f_{GC}}{2}) =
0.06234375$ \\
$U_{CA}=U_{CT}=U_{GA}=U_{GT}=(\frac{f_{GC}}{2})(\frac{f_{AT}}{2}) =
0.06234375$ \\
\end{center}

Where $f_{GC}=0.475$ and $f_{AT}=0.525$ are background frequencies of
G/C and A/T, respectively.  Then we derive emission probabilities for
the Homologous state as, for example:
\begin{equation}
\log_{2}\bigg(\frac{H_{AA}}{U_{AA}}\bigg) = \frac{91}{\psi},
\end{equation}
where $\frac{1}{\psi}$ is the unknown scaling factor used normalize $H_{CC}$ to 100. The full list of equations follows:
\begin{center}
$\log_{2}\bigg(\frac{H_{AC}}{U_{AC}}\bigg) = \frac{-114}{\psi},$
$\log_{2}\bigg(\frac{H_{AG}}{U_{AC}}\bigg) = \frac{-31}{\psi}$ \\
$\log_{2}\bigg(\frac{H_{AT}}{U_{AC}}\bigg) = \frac{-123}{\psi},$
$\log_{2}\bigg(\frac{H_{CG}}{U_{CC}}\bigg) = \frac{-125}{\psi}$ \\
$\log_{2}\bigg(\frac{H_{AA}}{U_{AA}}\bigg) = \ \ \ \frac{91}{\psi},$
$\log_{2}\bigg(\frac{H_{CC}}{U_{CC}}\bigg) = \ \ \ \frac{100}{\psi}$ \\
\end{center}

The system of six equations has seven free variables.  Moreover, the $H_{xy}$ must sum to 1 to make a probability distribution:
\begin{equation}
H_{AA} + H_{AC} + H_{AG} + H_{AT} + H_{CC} + H_{CG} = 1
\end{equation}
%$0.03072937146$
%$32.54215601$
We can solve the above six equations for $H_{xy}$ and substitute the
resulting expressions in to the normalizing equation to solve for
$\psi$. For the HOXD matrix the scaling factor is $\psi=32.5421$. Given
$\psi$, we can calculate values for all $H_{xy}$.

Emission probabilities in the Unrelated state can be directly adapted to the
background nucleotide distribution as follows:
\begin{center}
$U_{AA}=U_{AT}=U_{TA}=U_{TT}=(\frac{f_{AT}}{2})(\frac{f_{AT}}{2})$
$U_{CC}=U_{CG}=U_{GC}=U_{GG}=(\frac{f_{GC}}{2})(\frac{f_{GC}}{2})$
$U_{AC}=U_{AG}=U_{TC}=U_{AG}=(\frac{f_{AT}}{2})(\frac{f_{GC}}{2})$
$U_{CA}=U_{CT}=U_{GA}=U_{GT}=(\frac{f_{GC}}{2})(\frac{f_{AT}}{2})$
\end{center}
where the notation $U_{XY}$ indicates the probability of observing nucleotide X aligned to nucleotide Y in Unrelated sequence.  $f_{AT}$ is the fraction of nucleotides which are A/T and $f_{GC}$ is the fraction of G/C in the input genome.  Note that $f_{AT}=1-f_{GC}$.

\section*{Adapting HMM emission frequencies from an arbitrary substitution matrix to organisms with different G+C content}
HMM probabilities can be derived using any strand/species symmetric nucleotide substitution matrix,
but any particular matrix makes specific assumptions about divergence time, mutation pressures,
and sequence composition of the aligned sequences.
Genomes can range in G+C content from 30-75\%, and at the extremes,
a substitution matrix derived on 47.5\% GC sequence (such as HOXD) does not
perform well.  We have thus developed a method to adapt HMM emission
frequencies derived from an arbitrary substitution matrix
to organisms with different G+C content

Emission
probabilities in the Unrelated state can be directly adapted to the
background nucleotide distribution as follows:
\begin{center}
$U_{AA}=U_{AT}=U_{TA}=U_{TT}=(\frac{f_{AT}}{2})(\frac{f_{AT}}{2})$
$U_{CC}=U_{CG}=U_{GC}=U_{GG}=(\frac{f_{GC}}{2})(\frac{f_{GC}}{2})$
$U_{AC}=U_{AG}=U_{TC}=U_{AG}=(\frac{f_{AT}}{2})(\frac{f_{GC}}{2})$
$U_{CA}=U_{CT}=U_{GA}=U_{GT}=(\frac{f_{GC}}{2})(\frac{f_{AT}}{2})$
\end{center}
where the notation $U_{XY}$ indicates the probability of observing nucleotide X aligned to
nucleotide Y in Unrelated sequence.  $f_{AT}$ is the fraction of nucleotides which are A/T and
$f_{GC}$ is the fraction of G/C in the input genome.  Note that $f_{AT}=1-f_{GC}$.

To adapt emission probabilities in the homologous state, we consider the
probability of nucleotide substitution under the standard assumption
that sequences evolve according to a continuous-time reversible Markov process.
Thus, the probability of nucleotide $X$ mutating to nucleotide $Y$ over time $t$
can be written as: $H_{XY}=\pi_X Q(X,Y)t$, where $\pi_X$ represents the background
frequency of nucleotide $X$ and $Q(X,Y)$ is the instantaneous substitution rate for $X$
mutating to $Y$. Because the background frequencies on which the substitution matrix was computed are
known (e.g. 47.5\%GC for HOXD), we can simply replace the $\pi_X$ from the HOXD matrix
with the $\pi'_X$ from the input genome. Given that the substitution process is
strand-symmetric and reversible, we can compute adapted
emission probabilities for the Homologous state as follows:
\begin{center}
$H'_{AG}=H_{AG}$, \ \ \ $H'_{AC}=H_{AC}$ \\
$H'_{AA}=(\frac{f'_{AT}}{f_{AT}})H_{AA}$, \ \ \ \
$H'_{AT}=(\frac{f'_{AT}}{f_{AT}})H_{AT}$\\
$H'_{CC}=(\frac{f'_{GC}}{f_{GC}})H_{CC}$, \ \ \ \
$H'_{CG}=(\frac{f'_{GC}}{f_{GC}})H_{CG}$\\
\end{center}
where $f'_{AT}$ and $f'_{GC}$ represent the input genome's AT and GC content,
$H_{XY}$ values are the original emission probabilities derived from the
substitution matrix, and $H'_{XY}$ are emission probabilities adapted to
the input genome's G+C content.

\bibliographystyle{splncs}
\bibliography{procrastination}
\end{document}

