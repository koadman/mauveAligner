\subsection*{Calculating emission frequencies from the HOXD70 substitution matrix}
We calculated the emission frequencies in the
Homologous and Unrelated state using the same equation used to
calculate the values of the HOXD70 substitution matrix on 47.5\%G+C
content sequence~\cite{hoxd}:
\begin{equation}
s(x,y)= \log_{2}{\Bigg(\frac{p(x,y)}{q_{1}(x)q_{2}(y)}\Bigg)}
\end{equation}
{w}here $p(x,y)$ is the fraction of the observed aligned pairs of
nucleotides $x$ and $y$ in the training set used and $q_{1}(x)$ and
$q_{2}(y)$ denote the background frequencies of $x$ and $y$,
respectively. Chiaromonte \textit{et al.} scaled the resulting
$s(x,y)$ values so the largest was 100,
with the rest rounded to the nearest integer.  Given that the training
data has $47.5\%$~$G+C$ content and considering reversibility of the substitution process, we can compute emission probabilities for the Unrelated state
of our HMM:
\begin{multline}\\
U_{AA}=U_{AT}=U_{TA}=U_{TT}=(\frac{f_{AT}}{2})(\frac{f_{AT}}{2})
= 0.06890625 \\
U_{CC}=U_{CG}=U_{GC}=U_{GG}=(\frac{f_{GC}}{2})(\frac{f_{GC}}{2}) =
0.05640625 \\
U_{AC}=U_{AG}=U_{TC}=U_{TG}=(\frac{f_{AT}}{2})(\frac{f_{GC}}{2}) =
0.06234375 \\
U_{CA}=U_{CT}=U_{GA}=U_{GT}=(\frac{f_{GC}}{2})(\frac{f_{AT}}{2}) =
0.06234375 \\
\end{multline}
Where $f_{GC}=0.475$ and $f_{AT}=0.525$ are background frequencies of
G/C and A/T, respectively and the notation $U_{XY}$ indicates the probability of observing nucleotide X aligned to
nucleotide Y in Unrelated sequence.    Note, each $f_{XY}$ is divided by 2 to yield the frequency of a single nucleotide from dinucleotide background frequencies.

Previous work
has shown how to extract scaling factors and background frequencies from substitution matrices~\cite{ref-rev3a}. Accordingly, we derive emission probabilities for
the Homologous state as, for example:
\begin{equation}
\log_{2}\bigg(\frac{H_{AA}}{U_{AA}}\bigg) = \frac{91}{\psi},
\end{equation}
where $\frac{1}{\psi}$ is the unknown scaling factor used to normalize $H_{CC}$ to 100. The full list of equations follows:
\begin{multline}\\
\log_{2}\bigg(\frac{H_{AC}}{U_{AC}}\bigg) = \frac{-114}{\psi}\\
\log_{2}\bigg(\frac{H_{AG}}{U_{AC}}\bigg) = \frac{-31}{\psi}\\
\log_{2}\bigg(\frac{H_{AT}}{U_{AC}}\bigg) = \frac{-123}{\psi}\\
\log_{2}\bigg(\frac{H_{CG}}{U_{CC}}\bigg) = \frac{-125}{\psi}\\
\log_{2}\bigg(\frac{H_{AA}}{U_{AA}}\bigg) = \frac{91}{\psi}\\
\log_{2}\bigg(\frac{H_{CC}}{U_{CC}}\bigg) = \frac{100}{\psi}\\
\end{multline}

The system of six equations has seven free variables.  Moreover, the $H_{xy}$ must sum to 1 to make a probability distribution:
\begin{equation}
2H_{AA} + 4H_{AC} + 4H_{AG} + 2H_{AT} + 2H_{CC} + 2H_{CG} = 1
\end{equation}
%$0.03072937146$
%$32.54215601$
We can solve the above six equations for $H_{xy}$ and substitute the
resulting expressions in to the normalizing equation to solve for
$\psi$. For the HOXD70 matrix the scaling factor is $\psi=66.667$. Given
$\psi$, we can calculate values for all $H_{xy}$.

\subsection*{Adapting HMM parameters to organisms with different G+C content}

Emission
probabilities in the Unrelated state can be directly adapted to the
background nucleotide distribution ($f_{GC}, f_{AT}$ using \textbf {Equation 4}. To adapt emission probabilities in the homologous state, we consider the
probability of nucleotide substitution under the standard assumption
that sequences evolve according to a continuous-time reversible Markov process.
Thus, the probability of nucleotide $X$ mutating to nucleotide $Y$ over time $t$
can be written as: 
\begin{equation}
H_{XY}=\pi_X \exp^{Q(X,Y)t}
\end{equation}
, where $\pi_X$ represents the background
frequency of nucleotide $X$ and $Q(X,Y)$ is the instantaneous substitution rate for $X$
mutating to $Y$. Important to note is that we make the assumption that the probability of
either nucleotide at the root node is equal to their background
frequencies in the sequence. Also, as do other existing methods that correct for composition bias using an adapted substitution matrix~\cite{repseek}, we explicitly assume $t$ to be 1.  Because the background frequencies on which the substitution matrix was computed are
known (e.g. 47.5\%GC for HOXD70), we can replace the $\pi_X$ from the HOXD70 matrix
with the $\pi'_X$ from the input genome as follows:

\begin{equation}
H'_{XY}=(\frac{\pi'_X}{\pi_X})  H_{XY}
\end{equation}

%DERIVATION HERE!!
Given that the substitution process is
strand-symmetric and reversible, we can thus compute adapted
emission probabilities for the Homologous state as follows:
\begin{multline}\\
H'_{AG}=H_{AG}\\
H'_{AC}=H_{AC}\\
H'_{AA}=(\frac{f'_{AT}}{f_{AT}})H_{AA}\\
H'_{AT}=(\frac{f'_{AT}}{f_{AT}})H_{AT}\\
H'_{CC}=(\frac{f'_{GC}}{f_{GC}})H_{CC}\\
H'_{CG}=(\frac{f'_{GC}}{f_{GC}})H_{CG}\\
\end{multline}
where $f'_{AT}$ and $f'_{GC}$ represent the input genome's AT and GC content,
$H_{XY}$ values are the original emission probabilities derived from the
substitution matrix, and $H'_{XY}$ are emission probabilities adapted to
the input genome's G+C content.


%\section{Configuring the transition probabilities of the HMM}
